\section{Used technologies}
\subsection{Device}
\subsubsection{Python}
Libraries
\begin{enumerate}
 \item python-serial
 \item SocketIO
\end{enumerate}
\paragraph{Threads}
The application on the Raspberry Pi uses threads so the application can simultaneously read data from the Serial Interface to the Arduino and send data to the local database or the server.
\subsubsection{MySQL}
When the application can't send data to the remote server, all data is stored on a local MySQL database. MySQL is an open source relational database engine.
\subsubsection{Web sockets}
\subsubsection{Raspberry Pi}
\subsubsection{Arduino Nano}
Libraries
\begin{enumerate}
 \item INSERT LIBRARIES
\end{enumerate}

\subsection{Web interface}
\subsubsection{Javascript}
Javascript is a programing language used to program the behavior of a web page. It is easy to implement in a HTML file. An advantage of Javascript is that it is executed on the client side. This way, the web server won't unnecessarily be strained. A disadvantage is that a lot of lines have to be written in order to get some basic functionality. 
In this project, Javascript is used to create all the functionality on the client site. Without Javascript, the client site would not be able to get any data from the server and would thus be useless. 

\subsubsection{JQuery}
Jquery is a JavaScript library, created to simplify JavaScript programming. Using JQuery, lots of commands are simplified. In almost every JavaScript document used in this project, JQuery commands are present. 
For the calendar, an external Jquery library 'DataTables' was used. 
\textbf{RUGEN schrijf hier nog bij}

\subsubsection{JSON}
JSON is a data interchange format. Although it stands for JavaScript Object Notation, it is language independent. JSON is relatively easy to understand and is faster than other similar formats. In this project, JSON is used for data interchange between the client site and the data bank. 
